\documentclass[11pt]{article}
\usepackage[italian]{babel}
\usepackage{geometry}
\usepackage{pifont}
\usepackage{graphicx} 
\usepackage{titlesec}
\geometry{a4paper} 
\usepackage{listings} % necessario per inclusione codice sorgente
\usepackage{color} % syntax highlighting
\usepackage{url}
\usepackage[export]{adjustbox}
% definizione dei colori  
\definecolor{dkgreen}{rgb}{0,0.6,0}
\definecolor{gray}{rgb}{0.5,0.5,0.5}
\definecolor{mauve}{rgb}{0.58,0,0.82}

\pdfinfo{
   /Author (Andrei Ciulpan)
   /Title  (Specifica del Progetto di Laboratorio - Architetture II - Turno A)
}
\title{Specifica del Progetto di Laboratorio \\ Architettura degli Elaboratori II}
\author{Andrei Ciulpan\\ Matricola: 872394 - Turno: A\\ \url{andrei.ciulpan@studenti.unimi.it}}
\date{}

\begin{document}
\maketitle

\section{Introduzione}

Il progetto consiste nella scrittura di un programma in linguaggio assembly utilizzando il simulatore MARS.
Il programma chiederà all'utente l'introduzione di un numero via console, dopodichè
esporrà un menu cosi' come segue:

\begin{enumerate}
\item Calcola la sequenza di Collatz del numero N
\item Stabilisci se N è primo 
\item Stabilisci se N è pari/dispari 
\item Dammi un numero casuale compreso tra 1 e N
\item Riprova con un altro numero
\item Esci
\end{enumerate}


L'output del programma dipenderà dall'opzione scelta.

Saranno presenti le seguenti funzioni:  

\begin{enumerate}
\item int collatz(int N) \# funzione ricorsiva
\item int primo(int N))
\item int pari(int N)
\item int random(int N)

\end{enumerate}



\end{document}
